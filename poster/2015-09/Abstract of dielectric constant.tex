%% AMS-LaTeX Created with the Wolfram Language for Students - Personal Use Only : www.wolfram.com

\documentclass{article}
\usepackage{amsmath, amssymb, graphics, setspace}

\newcommand{\mathsym}[1]{{}}
\newcommand{\unicode}[1]{{}}

\newcounter{mathematicapage}
\begin{document}

\title{Electric-Field Mapped Averaging for Dielectric Constant}
\author{Weisong Lin. Andrew J Schultz. David A. Kofke}
\date{}
\maketitle
Department of Chemical and Biological Engineering, University at Buffalo, SUNY

Email: weisongl@buffalo.edu, ajs42@buffalo.edu, kofke@buffalo.edu


The dielectric constant of a material under given conditions is the ratio of permittivity or capacitance to that of vacuum. Dielectric constant is very important in the application of capacitor. Dielectric constant also controls the refractive index of  optical fibers. It would be important to measure dielectric constant accurately and precisely. 

Harmonically mapped averaging is an efficient technique for evaluating the anharmonic energy/pressure in for crystal system. We aim to apply mapped averaging to evaluate the dielectric constant from molecular simulation. The system we study are rigid dipoles  interacting with an external field E. The dielectric constant is the second derivative of the free energy with respect to external field. One way to compute it is to fixed the orientation,which is variance of total dipole moment. However,Variance always has large errors. The other way is to do transformation to make configuration appropriate by accounting the changing of orientation accompany with the external field. We do the transformation based on the idea that the dipoles would likely to be align with increase of external field. We compare the results of these two approaches in both Hard-Sphere and Lennard-Jones systems,the mapped average prove to be much more precise.


    
Key words: dielectric constant, mapped average

\end{document}
